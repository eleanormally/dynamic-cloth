\documentclass[11pt]{article}
\usepackage{amsmath} 
\usepackage{graphicx}
\usepackage{amssymb}
\usepackage{tikz}
\makeatletter
\renewcommand*\env@matrix[1][*\c@MaxMatrixCols c]{%
\hskip -\arraycolsep
\let\@ifnextchar\new@ifnextchar
\array{#1}}
\makeatother
\author{Eleanor Olson}
\title{Calculating a Feasible Parabolic Hanging Midpoint}

\begin{document}

\maketitle
Consider the following problem: Given a string of length $l$, let us approximate the angle that $l$'s midpoint will create with its endpoints $A$ and $B$, ($A, B \in \mathbb{R}^2$). This $l$ may further be defined as a scalar input $k$, multiplied by the distance of 2 original points. As such, we may use $l$ throughout this problem as a stand in for:\\
\indent $l=k\cdot ||M-N||$\\
Where $M$ and $N$ are points in $\mathbb{R}^2$.\\\\
To approximate the position of the point along their midpoint, let us think of the hanging function as a parabola. While this is not physically accurate, it will be easier to calcuate and can rougly approximate the physically correct catenary. We will also assume the following:
\indent\begin{itemize}
\item The string of length $l$ is perfectly inelastic
\item The positions of $A$ and $B$ will be close enough such that $||A-B|| \leq k\cdot ||M-N||$. If this is not the case, we may consider the angle a perfect $\pi$.
\end{itemize}
\newpage
\section{Possible Parabolas}
First, let us define the set of all parabolas that will contain $A$ and $B$. Let us define our parabola $f(x) = ax^2+bx+c$.\\\\
$\begin{aligned}
f(A_x) = A_y &= aA_x^2+bA_x+c\\
f(B_x) = B_y &= aB_x^2+bB_x+c\\\\
A_y-B_y &= a(A_x^2-B_x^2)+b(A_x-B_x)\\
&= a(A_x+B_x)(A_x-B_x)+b(A_x-B_x)\\
\frac{A_y-B_y}{A_x-B_x} &= a(A_x+B_x)+b\\\\
b &= \frac{A_y-B_y}{A_x-B_x}-a(A_x+B_x)\\\\\
A_y&=aA_x^2+\left(\frac{A_y-B_y}{A_x-B_x}-a(A_x+B_x)\right)A_x +c\\
c &= A_y - aA_x^2+\left(\frac{A_y-B_y}{A_x-B_x}-a(A_x+B_x)\right)A_x\\\\\\
f(x) &= ax^2+\left(\frac{A_y-B_y}{A_x-B_x}-a(A_x+B_x)\right)x+A_y - aA_x^2+\left(\frac{A_y-B_y}{A_x-B_x}-a(A_x+B_x)\right)A_x
\end{aligned}$\\\\\\
we now have an explicit equation for our parabola given we can solve for $a$. Now, let us determine the set of possible $a$'s through our provided $l$.
\newpage
\section{Arc Length}
We can define the arc length of our parabola as:\\
$\int \limits_{A_x}^{B_x}\sqrt{1+f'(x)^2}dx = l$\\\\
From above, we can evaluate:\\
$f'(x) = 2ax+\left(\frac{A_y-B_y}{A_x-B_x}-a(A_x+B_x)\right) \\
\indent\to \\
\int \limits_{A_x}^{B_x}\sqrt{1+2ax+\left(\frac{A_y-B_y}{A_x-B_x}-a(A_x+B_x)\right)}dx = l\\\\
\left(\frac{\left(-a(A_x+B_x)+2ax+\frac{A_y-B_y}{A_x-B_x}+1\right)^\frac{3}{2}}{3a}\right) \biggr\rvert_{x=A_x}^{x=B_x}=l\\\\
\frac{\left(-a(A_x+B_x)+2aB_x+\frac{A_y-B_y}{A_x-B_x}+1\right)^\frac{3}{2}-\left(-a(A_x+B_x)+2aA_x+\frac{A_y-B_y}{A_x-B_x}+1\right)^\frac{3}{2}}{3a}= l\\
$\\\\
To further simplify, let us set the position of $A$ to be $(0,0)$ (this is perfectly reasonable as we are projecting into $\mathbb{R}^2$ anyways). Therefore, we have the equation\\
$\frac{\left(-aB_x+2aB_x+\frac{B_y}{B_x}+1\right)^\frac{3}{2}-\left(-aB_x+\frac{B_y}{B_x}+1\right)^\frac{3}{2}}{3a}= l$\\\\
$\frac{\left(aB_x+\frac{B_y}{B_x}+1\right)^\frac{3}{2}-\left(-aB_x+\frac{B_y}{B_x}+1\right)^\frac{3}{2}}{3a}= l$\\\\
\\\\
Given this simplification, we may also simplify our definition of $b$ and $c$ to:\\
\indent $b = \frac{B_y}{B_x}-a(B_x)$\\
\indent $c = 0$\\\\
To properly solve this, we must simply plug in our values for $B_x, B_y,$ and $l$, then solve for $a$ numerically. Finally, we can plug our midpoint in and get the position, from which it is trivial to calculate its angle.
\end{document}
